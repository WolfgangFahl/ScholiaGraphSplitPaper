% SWAT4LS 2026 - Wikidata in healthcare and life sciences
% Integration and Tools: Exploring how Wikidata integrates with health data systems
% Real-World Applications: Showcasing case studies of Wikidata's impact
% Challenges: Identifying challenges in using Wikidata, with proposed solutions

%% The first command in your LaTeX source must be the \documentclass command.
\documentclass{ceurart}

%%
%% One can fix some overfulls
\sloppy

%%
%% Minted listings support 
\usepackage{listings}
\lstset{breaklines=true}

%%
%% Define ORCID macro if not defined by common input or class
\newcommand{\orcidID}[1]{\orcid{#1}}

%%
%% Rights management information.
\copyrightyear{2026}
\copyrightclause{Copyright for this paper by its authors.
  Use permitted under Creative Commons License Attribution 4.0
  International (CC BY 4.0).}

%%
%% This command is for the conference information
\conference{SWAT4LS 2026: Semantic Web Applications and Tools for Health Care and Life Sciences, 2026}

%%
%% The "title" command
\title{Scholia Wikidata Graph Split Mitigation}

%%
%% The "author" command and its associated commands
\author[1]{Egon L. Willighagen}[%
    orcid=0000-0001-7542-0286,
    % email=e.willighagen@maastrichtuniversity.nl, % Add email if desired
]

\author[2]{Wolfgang Fahl}[%
    orcid=0000-0002-0821-6995,
]

%\author[3,4,5]{Daniel Mietchen}[%
%    orcid=0000-0001-9488-1870,
%]

%\author[6]{Peter Patel-Schneider}[%
%    orcid=0000-0002-3563-3693,
%]

%\author[7]{Konrad Linden}[%
%    orcid=0000-0003-1321-2993,
%]

%% Affiliations
\address[1]{Department of Translational Genomics, NUTRIM, Maastricht University, Netherlands}
\address[2]{BITPlan GmbH, Willich, Germany}
%\address[3]{FIZ Karlsruhe -- Leibniz Institute for Information Infrastructure, Germany}
%\address[4]{Leibniz Institute of Freshwater Ecology and Inland Fisheries (IGB), Germany}
%\address[5]{Institute for Globally Distributed Open Research and Education (IGDORE), Germany}
%\address[6]{Institution Name Here}
%\address[7]{Institution Name Here}

\begin{document}

%%
%% The abstract
\begin{abstract}
Wikidata was forced to perform a graph split in 2025 which will lead to the deactivation of the full legacy graph server in January 2026. The split has forced the Scholia project to overhaul the backend and query set being in use.
\end{abstract}

%%
%% Keywords
\begin{keywords}
Scholia \sep 
Wikidata
\end{keywords}

%%
%% This command processes the author and affiliation and title information
\maketitle

\section{Introduction}
The Wikidata graph split \cite{wikidata:wdqs:graphsplit} has challenged the continuation of the Scholia service. 

\section{Background}
Scholia \cite{Nielsen_Scholia_Scientometrics_and_2017} is a portal for the scientific community that
provides access to Wikidata \cite{Vrandecic2014} curated content using a set of 387 queries organized in over 20 aspects. General aspects such as Author, Work, Venue, Topic, Organization, Publisher, Event and Location are covered as well as biochemical related ones such as Chemical compound, Gene/Protein, Disease and Taxon.

The Wikimedia Foundation provided Wikidata Query Service is using blazegraph as its SPARQL endpoint. The blazegraph 4TB storage limit is a threat to the continuation of the Wikidata service. Given the close to exponential growth of content since the launch in October 2012 the limit would be met within the upcoming months. Therefore the \cite{wikidata:wdqs:graphsplit} graph split was announced and performed in 2025. 
The split led to having a wikidata-main graph with 8.6 billion triples and a wikidata-scholia graph with 8.7 billion triples instead of the full graph havin 17.1 billion triples.

\section{Options}
Given the deadline of January 2026 at which point the full graph provided by the Wikimedia Foundation will not be available any more the Scholia project was forced to take action and pursued the following options:
\begin{itemize}
   \item modify affected queries to use federation accross the two split graphs 
  \item migrate to a different backend such as QLever
\end{itemize}


\section{Exploring Backend options}

%%
%% Define the bibliography file to be used
%% Ensure you have a 'references.bib' file in the same directory
\bibliography{references}

\end{document}