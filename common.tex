%% common LaTeX set up
%% WF 2025-12-06 (revised for pdfLaTeX/LNCS with UTF-8 workarounds)

% UTF-8 input and font encoding
%\usepackage[utf8]{inputenc}
%\usepackage[T1]{fontenc}  % Standard encoding for better character support

% Standard LNCS fonts
%\usepackage{mathptmx}  % Times-like for math/text
%\usepackage{helvet}    % Helvetica for sans-serif
%\usepackage{courier}   % Courier for monospace (limited UTF-8, hence workarounds below)

% Multicolumn support
%\usepackage{multicol}

% Linenumbers (if needed; comment out otherwise)
\usepackage{lineno}

% Make references clickable (load late to avoid conflicts)
\usepackage{hyperref}

% Source code rendering (listings for simplicity)
\usepackage{listings}
\usepackage{xcolor}  % For listings colors

% Graphics and SVG (if needed; SVG requires inkscape or similar with --shell-escape)
\usepackage{graphicx}
\usepackage{svg}

%% Custom Commands
% ORCID ID
\providecommand{\orcidID}[1]{%
  \href{https://orcid.org/#1}{%
    \includegraphics[width=8pt]{images/orcid.png}%
  }%
}
\newcommand\wf{{Wolfgang Fahl \orcidID{0000-0002-0821-6995}}}

% Footnotelink
\newcommand{\footnotelink}[2]{\href{#1}{#2}\footnote{#1}}
\newcommand{\footnotelonglink}[2]{#2\footnote{\url{#1}}}

% Typographical quoting
\newcommand{\tquote}[1]{``#1''}

% Wikidata references
\newcommand{\wdq}[2]{\footnotelink{https://www.wikidata.org/wiki/#2}{#1 (#2)}}
\newcommand{\wdp}[2]{\footnotelink{https://www.wikidata.org/wiki/Property:#2}{#1 (#2)}}

% In-document referencing
\newcommand{\seeref}[1]{(see Section~\ref{#1})}
\newcommand{\fullref}[1]{\ref{#1}.~\nameref{#1}}

% Listings setup with UTF-8 workarounds (replace unsupported chars like emojis)
\lstset{
  basicstyle=\footnotesize\ttfamily,
  breaklines=true,
  frame=single,
  numbers=left,
  numberstyle=\tiny\color{gray},
  keywordstyle=\color{blue},
  commentstyle=\color{green},
  stringstyle=\color{red},
  extendedchars=true,  % Basic UTF-8 support
  literate={��}{{[office]}}1  % Workaround: Replace emoji with text (add more as needed, e.g., {��}{{[smile]}}1)
}

% YAML language definition for listings
\lstdefinelanguage{yaml}{
  keywords={true,false,null,y,n},
  keywordstyle=\color{blue}\bfseries,
  ndkeywords={label, wikidata_id, icon, type, web_url, from, to, sequence, start_node},
  ndkeywordstyle=\color{darkgray}\bfseries,
  identifierstyle=\color{black},
  sensitive=false,
  comment=[l]{\#},
  morecomment=[s]{/*}{*/},
  commentstyle=\color{purple}\ttfamily,
  stringstyle=\color{red}\ttfamily,
  morestring=[b]',
  morestring=[b]"
}

% Example usage in your main.tex:
% \begin{lstlisting}[language=yaml]
% key: value ��  % This will render as "key: value [office]"
% \end{lstlisting}